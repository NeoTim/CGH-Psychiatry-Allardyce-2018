\documentclass{beamer}
\usepackage[no-math]{fontspec}
\usepackage{xeCJK}
\setCJKmainfont{Source Han Sans TW}
\hypersetup{colorlinks,linkcolor=}

\usetheme{CambridgeUS}
\title[(Allardyce \textit{et al}, 2018)]{Association between schizophrenia-related polygenic liability and the occurrence and level of mood-incongruent psychotic symptoms in bipolar disorder}
\subtitle{Allardryce \textit{et al}, \textit{J Am Med Assoc Psychiatry}, 2018}
\author[Chen-Pang He]{何震邦 (Chen-Pang He), Intern}
\date{August 9, 2018}
\institute[CGH]{Cathay General Hospital}

\newcommand*{\solo}[1]{\centering\includegraphics[width=\textwidth, height=0.8\textheight, keepaspectratio]{#1}}

\begin{document}
\maketitle

\section{Introduction}
\begin{frame}{Introduction}
    \begin{itemize}
        \item Bipolar disorder (BD) overlaps considerably with schizophrenia
              (SCZ) in both its clinical presentation and genetic liability.
        \item Bipolar disorder is a phenomenologically heterogeneous construct.
        \item It has been proposed that
            \begin{itemize}
                \item This clinical heterogeneity indicates underlying etiological heterogeneity.
                \item The degree of clinical similarity between BD and SCZ reflects overlapping alleles.
            \end{itemize}
    \end{itemize}
\end{frame}

\begin{frame}{Mood congruence of psychotic features}
    \begin{itemize}
        \item Delusions and hallucinations are common in BD
            \begin{itemize}
                \item 1/3 judged to be mood incongruent
            \end{itemize}
        \item Mood-incongruent psychotic features are associated with
            \begin{itemize}
                \item Poor prognosis
                \item Poor lithium response
                \item Qualitatively similar to the prototypic symptoms of SCZ
            \end{itemize}
    \end{itemize}

    This suggests that BD with psychosis and particularly mood-incongruent
    psychotic features has stronger etiological links to SCZ.
\end{frame}

\begin{frame}{Genome-wide association studies (GWAS)}
    \begin{itemize}
        \item GWAS have found a substantial polygenic component to both BD and SCZ risks.
        \item This risk can be calculated with the polygenic risk score (PRS).
            \begin{itemize}
                \item Higher scores indicate a higher burden of risk alleles.
            \end{itemize}
        \item The PRS-SCZ differentiates BD cases from controls.
        \item Schizoaffective bipolar disorder (SABD) has a relatively larger burden of SCZ risk, compared with other BD subtypes.
    \end{itemize}
\end{frame}

\begin{frame}{Hypotheses}
    \begin{itemize}
        \item BD with psychosis would be associated with higher polygenic risk for SCZ.
        \item This association would be stronger when mood-incongruent psychotic features were present.
    \end{itemize}
\end{frame}

\section{Methods}
\subsection{Sample ascertainment}
\begin{frame}{Bipolar disorder sample}
    \begin{itemize}
        \item 4436 cases of bipolar disorder
            \begin{itemize}
                \item Deep phenotypic information
                \item European ancestry and domicile in the United Kingdom
                \item Data collected in 2000--2013
            \end{itemize}
        \item Stratification
            \begin{itemize}
                \item Subtypes in Research Diagnostic Criteria (RDC)
                    \begin{itemize}
                        \item Bipolar I (BD I)
                        \item Bipolar II (BD II)
                        \item Schizoaffective bipolar disorder (SABD)
                    \end{itemize}
                \item Lifetime ever occurrence of psychotic symptoms (LEP)
                \item Level of mood incongruence (LMI)
            \end{itemize}
    \end{itemize}

    RDC differentiates individuals on the basis of their pattern of mood and
    psychotic symptoms better than DSM-5 and ICD-10.
\end{frame}

\begin{frame}{Schizophrenia sample}
    \begin{itemize}
        \item 4976 cases of the CLOZUK study sample
            \begin{itemize}
                \item Treatment-resistant schizophrenia, treated with clozapine
            \end{itemize}
    \end{itemize}

    In principle, treatment-resistant SCZ may carry higher polygenic risk
    burden; however, the PRSs in the CLOZUK sample are similar to the PRSs in
    other SCZ samples used by the Psychiatric Genomics Consortium.
\end{frame}

\begin{frame}{Control samples}
    The controls came from 2 UK sources:
    \begin{itemize}
        \item 2532 cases from Type 1 Diabetes Genetics Consortium
            \begin{itemize}
                \item Unscreened
            \end{itemize}
        \item 6480 cases from Generation Scotland
            \begin{itemize}
                \item Screened for psychiatric disorders
            \end{itemize}
    \end{itemize}
\end{frame}

\begin{frame}{Principal components analysis}
    \begin{itemize}
        \item To adjust for potential confounding from population structure, principal components analysis was performed.
        \item The eigenvectors for the first 10 principal components were kept to use as covariates in the association analysis.
    \end{itemize}
\end{frame}

\begin{frame}{Polygenic risk scores}
    PRSs were generated using the 2014 Psychiatric Genomics Consortium SCZ
    meta-analysis as the discovery set.
\end{frame}

\begin{frame}{Statistical analysis}
    A multinomial logit model was used to estimate differential associations of
    standardized PRSs, adjusted for the first 10 principal components and
    genotyping platforms across the categories of cases and controls.
\end{frame}

\section{Results}
\begin{frame}{Figure 1}
    \solo{F1.eps}
\end{frame}

\begin{frame}{Figure 2}
    \solo{F2.eps}
\end{frame}

\begin{frame}{Figure 3}
    \solo{F3.eps}
\end{frame}

\begin{frame}{Table 1}
    \solo{T1.eps}
\end{frame}

\begin{frame}{Table 2}
    \solo{T2.eps}
\end{frame}

\section{Discussion}
\begin{frame}{Discussion}
    \begin{itemize}
        \item Higher PRS-SCZ in BD is well established.
        \item This observation is replicated and extended.
        \item There is a gradient of PRS associations across SCZ and BD subtypes,
            in descending order:
            \begin{itemize}
                \item CLOZUK
                \item SABD
                \item BD I with psychosis
                \item BD I without psychosis
                \item BD II
            \end{itemize}
    \end{itemize}
\end{frame}

\begin{frame}{Discussion (cont.)}
    \begin{itemize}
        \item Previously published work examining the PRSs for SCZ across BD,
              stratified by psychosis, did not find significant discrimination,
              although a trend was observed that is consistent with the
              findings presented here. 
        \item The most likely explanations: a larger SCZ-GWAS discovery set is
              used in this study, which reduces the measurement error and
              improves power from both this sample and the larger BD sample.
    \end{itemize}
\end{frame}

\begin{frame}{Discussion (cont.)}
    \begin{itemize}
        \item This group has shown that PRS-SCZ significantly differentiates
              SABD from non-SABD subtypes, while finding no statistically
              significant differential between BD stratified by psychosis,
              suggesting it is the nature of the psychotic symptoms rather than
              their presence that better indexes the liability shared with SCZ.
        \item The current analysis supports the proposition that it is the
              level of mood incongruence rather than the presence of psychosis
              that better specifies a shared biologically validated dimensional
              trait, which is captured, although with less precision, by the
              SABD diagnostic category.
    \end{itemize}
\end{frame}

\begin{frame}{Discussion (cont.)}
    \begin{itemize}
        \item Psychosis and mood-incongruent psychotic features are known to be
              correlated with poorer prognosis and treatment response.
        \item The PRS derived from SCZ-GWAS may be indexing a general liability
              for psychopathological severity (at least in part) rather than a
              (SCZ) disease-specific liability.
    \end{itemize}
\end{frame}

\subsection{Implications}
\begin{frame}{Implications}
    \begin{itemize}
        \item This study upports the hypothesis that, within BD, positive and
              disorganized psychotic symptoms represent a dimensionally defined
              stratum with underpinning biological validity.
            \begin{itemize}
                \item Particularly, mood-incongruent psychotic features
            \end{itemize}
        \item These features are not only phenotypically similar to those
              observed in prototypal SCZ but also index a greater
              shared-genetic liability.
            \begin{itemize}
                \item This suggests BD and SCZ share more pathophysiological features.
            \end{itemize}
    \end{itemize}
\end{frame}

\begin{frame}{Implications (cont.)}
    \begin{itemize}
        \item Notably, in those diagnosed with BD I with no history of
              psychosis, the association with SCZ liability was weaker but
              still higher than in the control group.
        \item There was no overlap with SCZ liability in the BD II subsample. 
        \item Stronger genetic links between the risk for SCZ and BD
              characterized by the occurrence of psychosis and level of mood
              incongruence has been established.
    \end{itemize}
\end{frame}

\subsection{Limitations}
\begin{frame}{Limitations}
    \begin{itemize}
        \item Phenotypic misclassification is a potential methodological concern.
        \item Cases and controls were collected independently, which can result
              in confounding due to population stratification and potential batch
              effects across the cases and controls.
        \item Only the effect of common variants is examined, as rare variants
              are not captured by current GWAS.
    \end{itemize}
\end{frame}

\section{Conclusions}
\begin{frame}{Conclusions}
    \begin{itemize}
        \item A gradient of polygenic liability across SCZ and BD, indexed by
              the occurrence and level of mood incongruence of positive and
              disorganized psychotic symptoms
        \item The usefulness of genetic data to dissect clinical heterogeneity
              within and across disorders
        \item Aid in defining patient stratifiers with improved biological
              precision and validity
        \item Moving tentatively toward precision medicine in psychiatry
    \end{itemize}
\end{frame}
\end{document}
