\documentclass{beamer}
\usepackage[no-math]{fontspec}
\usepackage{xeCJK}
\setCJKmainfont{Source Han Sans TW}
\hypersetup{colorlinks,linkcolor=}

\usetheme{CambridgeUS}
\title[(Allardyce \textit{et al}, 2018)]{Association between schizophrenia-related polygenic liability and the occurrence and level of mood-incongruent psychotic symptoms in bipolar disorder}
\subtitle{Allardryce \textit{et al}, \textit{J Am Med Assoc Psychiatry}, 2018}
\author[Chen-Pang He]{何震邦 (Chen-Pang He), Intern}
\date{August 9, 2018}
\institute[CGH]{Cathay General Hospital}

\newcommand*{\solo}[1]{\centering\includegraphics[width=\textwidth, height=0.8\textheight, keepaspectratio]{#1}}

\begin{document}
\maketitle

\section{Key points}
\subsection{Question}
\begin{frame}{Question}
    What is the association between schizophrenia-related polygenic liability
    and the occurrence and level of mood-incongruence of psychotic symptoms in
    bipolar disorder?
\end{frame}

\subsection{Findings}
\begin{frame}{Findings}
    In this case-control study involving
    \begin{itemize}
        \item 4436 cases of bipolar disorder
        \item 4976 cases of schizophrenia
        \item 9012 controls,
    \end{itemize}
    there was an exposure-response gradient of polygenic risk.
\end{frame}

\begin{frame}{Findings}
    From strongest to lowest association, there were
    \begin{itemize}
        \item Schizophrenia
        \item Bipolar disorder with prominent mood-incongruent psychotic features
        \item Bipolar disorder with mood-congruent psychotic features
        \item Bipolar disorder with no psychosis;
    \end{itemize}
    all differential associations were statistically significant.
\end{frame}

\subsection{Meaning}
\begin{frame}{Meaning}
    This study shows a gradient of genetic liability across schizophrenia and
    bipolar disorder, indexed by the occurrence of psychosis and level of mood
    incongruence.
\end{frame}

\section{Methods}
\subsection{Sample ascertainment}
\begin{frame}{Bipolar disorder sample}
    \begin{itemize}
        \item 4436 cases of bipolar disorder
            \begin{itemize}
                \item Deep phenotypic information
                \item European ancestry and domicile in the United Kingdom
                \item Data collected in 2000--2013
            \end{itemize}
        \item Stratification
            \begin{itemize}
                \item Subtypes in Research Diagnostic Criteria
                    \begin{itemize}
                        \item Bipolar I (BD I)
                        \item Bipolar II (BD II)
                        \item Schizoaffective, bipolar type (SABD)
                    \end{itemize}
                \item Lifetime ever occurrence of psychotic symptoms (LEP)
                \item Level of mood incongruence (LMI)
            \end{itemize}
    \end{itemize}
\end{frame}

\section{Results}
\begin{frame}{Table 1}
    \solo{T1.eps}
\end{frame}

\begin{frame}{Table 2}
    \solo{T2.eps}
\end{frame}

\begin{frame}{Figure 1}
    \solo{F1.eps}
\end{frame}

\begin{frame}{Figure 2}
    \solo{F2.eps}
\end{frame}

\begin{frame}{Figure 3}
    \solo{F3.eps}
\end{frame}
\end{document}
