\documentclass{beamer}
\usepackage[no-math]{fontspec}
\usepackage{xeCJK}
\setCJKmainfont{Source Han Sans TW}
\hypersetup{colorlinks,linkcolor=}

\usetheme{CambridgeUS}
\title[(Allardyce \textit{et al}, 2018)]{Association between schizophrenia-related polygenic liability and the occurrence and level of mood-incongruent psychotic symptoms in bipolar disorder}
\subtitle{Allardryce \textit{et al}, \textit{J Am Med Assoc Psychiatry}, 2018}
\author[Chen-Pang He]{何震邦 (Chen-Pang He), Intern}
\date{August 9, 2018}
\institute[CGH]{Cathay General Hospital}

\newcommand*{\solo}[1]{\centering\includegraphics[width=\textwidth, height=0.8\textheight, keepaspectratio]{#1}}

\begin{document}
\maketitle

\section{Introduction}
\begin{frame}{Introduction}
    \begin{itemize}
        \item Bipolar disorder (BD) overlaps considerably with schizophrenia
              (SCZ) in both its clinical presentation and genetic liability.
        \item Bipolar disorder is a phenomenologically heterogeneous construct.
        \item It has been proposed that
            \begin{itemize}
                \item This clinical heterogeneity indicates underlying etiological heterogeneity.
                \item The degree of clinical similarity between BD and SCZ reflects overlapping alleles.
            \end{itemize}
    \end{itemize}
\end{frame}

\begin{frame}{Mood incongruence of psychotic features}
    \begin{itemize}
        \item Delusions and hallucinations are common in BD
            \begin{itemize}
                \item 1/3 judged to be mood incongruent
            \end{itemize}
        \item Mood-incongruent psychotic features are associated with
            \begin{itemize}
                \item Poor prognosis
                \item Poor lithium response
                \item Qualitatively similar to the prototypic symptoms of SCZ
            \end{itemize}
    \end{itemize}

    This suggests that BD with psychosis and particularly mood-incongruent
    psychotic features has stronger etiological links to SCZ.
\end{frame}

\begin{frame}{Genome-wide association studies (GWAS)}
    \begin{itemize}
        \item GWAS have found a substantial polygenic component to both BD and SCZ risks.
        \item This risk can be calculated with the polygenic risk score (PRS).
            \begin{itemize}
                \item Higher scores indicate a higher burden of risk alleles.
            \end{itemize}
        \item The PRS-SCZ differentiates BD cases from controls.
        \item Schizoaffective bipolar disorder (SABD) has a relatively larger burden of SCZ risk, compared with other BD subtypes.
    \end{itemize}
\end{frame}

\begin{frame}{Hypotheses}
    \begin{itemize}
        \item BD with psychosis would be associated with higher polygenic risk for SCZ.
        \item This association would be stronger when mood-incongruent psychotic features were present.
    \end{itemize}
\end{frame}

\section{Methods}
\subsection{Sample ascertainment}
\begin{frame}{Bipolar disorder sample}
    \begin{itemize}
        \item 4436 cases of bipolar disorder
            \begin{itemize}
                \item Deep phenotypic information
                \item European ancestry and domicile in the United Kingdom
                \item Data collected in 2000--2013
            \end{itemize}
        \item Stratification
            \begin{itemize}
                \item Subtypes in Research Diagnostic Criteria
                    \begin{itemize}
                        \item Bipolar I (BD I)
                        \item Bipolar II (BD II)
                        \item Schizoaffective bipolar disorder (SABD)
                    \end{itemize}
                \item Lifetime ever occurrence of psychotic symptoms (LEP)
                \item Level of mood incongruence (LMI)
            \end{itemize}
    \end{itemize}
\end{frame}

\begin{frame}{Schizophrenia sample}
    \begin{itemize}
        \item 4976 cases of the CLOZUK study sample
            \begin{itemize}
                \item Treatment-resistant schizophrenia, treated with clozapine
            \end{itemize}
    \end{itemize}

    In principle, treatment-resistant SCZ may carry higher polygenic risk
    burden; however, the PRSs in the CLOZUK sample are similar to the PRSs in
    other SCZ samples used by the Psychiatric Genomics Consortium.
\end{frame}

\begin{frame}{Control samples}
    The controls came from 2 UK sources:
    \begin{itemize}
        \item 2532 cases from Type 1 Diabetes Genetics Consortium
            \begin{itemize}
                \item Unscreened
            \end{itemize}
        \item 6480 cases from Generation Scotland
            \begin{itemize}
                \item Screened for psychiatric disorders
            \end{itemize}
    \end{itemize}
\end{frame}

\section{Results}
\begin{frame}{Table 1}
    \solo{T1.eps}
\end{frame}

\begin{frame}{Table 2}
    \solo{T2.eps}
\end{frame}

\begin{frame}{Figure 1}
    \solo{F1.eps}
\end{frame}

\begin{frame}{Figure 2}
    \solo{F2.eps}
\end{frame}

\begin{frame}{Figure 3}
    \solo{F3.eps}
\end{frame}
\end{document}
